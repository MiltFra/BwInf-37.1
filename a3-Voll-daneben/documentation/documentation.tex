\documentclass[a4paper,10pt,ngerman]{scrartcl}
\usepackage{babel}
\usepackage[T1]{fontenc}
\usepackage[utf8x]{inputenc}
\usepackage[a4paper,margin=2.5cm,footskip=0.5cm]{geometry}

% Die nächsten drei Felder bitte anpassen:
\newcommand{\Aufgabe}{Aufgabe 3: Voll daneben} % Aufgabennummer und Aufgabennamen angeben
\newcommand{\TeamID}{00922}       % Team-ID aus dem PMS angeben
\newcommand{\TeamName}{Zweiundvierzig} % Team-Namen angeben
\newcommand{\Namen}{Franz Miltz} % Namen der Bearbeiter/-innen dieser Aufgabe angeben
 
% Kopf- und Fußzeilen
\usepackage{scrlayer-scrpage, lastpage}
\setkomafont{pageheadfoot}{\large\textrm}
\lohead{\Aufgabe}
\rohead{Team-ID: \TeamID}
\cfoot*{\thepage{}/\pageref{LastPage}}

% Position des Titels
\usepackage{titling}
\setlength{\droptitle}{-1.0cm}

% Für mathematische Befehle und Symbole
\usepackage{amsmath}
\usepackage{amssymb}

% Für Bilder
\usepackage{graphicx}

% Für Algorithmen
\usepackage{algpseudocode}

% Für Quelltext
\usepackage{listings}
\lstset{language=Python}
\usepackage{color}
\definecolor{mygreen}{rgb}{0,0.6,0}
\definecolor{mygray}{rgb}{0.5,0.5,0.5}
\definecolor{mymauve}{rgb}{0.58,0,0.82}
\lstset{
  keywordstyle=\color{blue},commentstyle=\color{mygreen},
  stringstyle=\color{mymauve},rulecolor=\color{black},
  basicstyle=\footnotesize\ttfamily,numberstyle=\tiny\color{mygray},
  captionpos=b, % sets the caption-position to bottom
  keepspaces=true, % keeps spaces in text
  numbers=left, numbersep=5pt, showspaces=false,showstringspaces=true,
  showtabs=false, stepnumber=2, tabsize=2, title=\lstname
}
\lstdefinelanguage{JavaScript}{ % JavaScript ist als einzige Sprache noch nicht vordefiniert
  keywords={break, case, catch, continue, debugger, default, delete, do, else, finally, for, function, if, in, instanceof, new, return, switch, this, throw, try, typeof, var, void, while, with},
  morecomment=[l]{//},
  morecomment=[s]{/*}{*/},
  morestring=[b]',
  morestring=[b]",
  sensitive=true
}

% Diese beiden Pakete müssen zuletzt geladen werden
%\usepackage{hyperref} % Anklickbare Links im Dokument
\usepackage{cleveref}

% Daten für die Titelseite
\title{\textbf{\Huge\Aufgabe}}
\author{\LARGE Team-ID: \LARGE \TeamID \\\\
	    \LARGE Team-Name: \LARGE \TeamName \\\\
	    \LARGE Bearbeiter/-innen dieser Aufgabe: \\ 
	    \LARGE \Namen\\\\}
\date{\LARGE\today}

\begin{document}

\maketitle
\tableofcontents

\vspace{0.5cm}

\section{Lösungsidee}
Das Problem kann in zwei Schritten gel\"ost werden. Die Eingaben (d.h. die
von den Spielern gew\"ahlten Zahlen) werden immer als sortierte Liste betrachtet.\\
\indent Zun\"achst muss festgestellt werden, was die Intervalle kosten, wenn man
genau eine Zahl in jenes Intervall setzt. Dabei ist klar, dass der Preis minimal
wird, wenn die eigene Zahl in der Mitte liegt.
 Dabei ist der Begriff Mitte vollkommen losgel\"ost von den
Werten, die die einzelnen Eingaben haben. Die Mitte ist
immer dort, wo sowohl links, als auch rechts, gleich viele Eingaben befinden. 
Bei einer ungeraden Anzahl von Werten ist meine eigene Zahl also gleich der
mittleren. Bei einer ungeraden Anzahl kann ich zwischen den beiden mittleren
oder irgendeiner Zahl dazwischen w\"ahlen. Allgemein kann ich also die Anzahl der
Eingaben im Intervall ganzzahlig dividieren, um die Mitte zu finden.\\
\indent Auf diese Weise kann nun f\"ur jedes Intervall $[a, b]$ ein Preis und
die damit verbundene Mitte errechnet werden. Wenn man beachtet, dass $a \leq b \leq n$,
ergeben sich $\frac{1}{2}n(n-1)$ Paare $(a, b)$. Wenn wir davon ausgehen, dass
ermitteln des Preises eines Segmentes in konstanter Laufzeit erfolgt, ist dieser
Teil des Problems in $O(n^2)$ l\"osbar.\\
\indent Anschlie\ss end m\"ussen diese Segmente m\"oglichst optimal zusammengesetzt
werden. Dazu kann man versuchen ein optimales Interval $[1, b]$ aus den Intervallen $[1, a-1]$
und $[a, b]$ zu ermitteln. Der hintere Teil wurde bereits, wie oben beschrieben
berechnet. Der vordere Teil ist ebenfalls bereits vorhanden, wenn man in der
richtigen Reihenfolge vorgeht. Denn jenes Intervall enth\"alt genau eine Zahl 
der Bank weniger, als das was insgesamt errechnet werden soll. Somit kann das Problem
der Reihe nach mit 1 bis 10 Zahlen gel\"ost werden. F\"ur jedes der $n$ Intervalle
ergeben sich also $n-2$ (beide Teilintervalle m\"ussen mindestens ein Element haben)
m\"ogliche Aufteilungen, von denen die beste gefunden werden muss. Die Preise der
Teilintervalle wurden bereits berechnet, sind also in konstanter Laufzeit verf\"ugbar.
Also l\"asst sich auch diese Problem in $O(n^2)$ l\"osen.\\
\indent Um dann die vollst\"andige Liste der zehn Zahlen zu erhalten, muss einfach
zur\"uckverfolgt, welche Intervalle f\"ur jeden Schritt hinzugekommen sind. Deren
Mitten sind dann das Ergebnis.

\section{Umsetzung}
Das Programm wurde f\"ur den Python 3.7.1 Interpreter geschrieben und
auf einem Linux-System getestet. Es sind keine weiteren Packages erforderlich.\\
\indent Es gibt zwei M\"oglichkeiten die Inputs an das Programm zu \"ubergeben:
Einerseits kann die Datei als Argument in der Command Line \"ubergeben
werden, andererseits kann das Programm auch mit dem Pipe-Operator 
(\texttt{<}) umgehen. Die Ausgabe ist stets der Preis und die Liste der
sogenannten Picks (Zahlen, die von der Bank gew\"ahlt werden, damit der Preis
minimal ist). Au\ss erdem \"uberpr\"uft das Programm, ob diese Picks bei dieser
Eingabe zu diesem Preis f\"uhren und teilt das Ergebnis mit.\\
\indent Wie oben beschrieben, kann die Probleml\"osung in zwei Teile gegliedert
werden. Analog dazu werden in der Klasse, die zur L\"osung des Problems genutzt
wird, \texttt{volldaneben\_solver}, nacheinander die Methoden
\texttt{set\_segment\_prices} (zum Berechnen der Einzelpreise aller Segmente)
und \texttt{set\_combined\_prices} (zum Berechnen der optimalen Verkettung
der Segmente) aufgerufen. Zuvor wird jedoch die Input-Liste sortiert.
\subsection{Einzelsegmentpreise}
Aus einer Liste von Eingaben zu ermitteln, was der minimale Preis ist, ist
einfach. Man w\"ahlt das mittlere Element und berechnet die Summe der 
Betr\"age der Differenzen der Elemente zu jener Mitte.
\begin{lstlisting}
def segment_price(self, segment):                                            
  pick = segment[len(segment) // 2]                                           
  return sum([abs(x - pick) for x in segment])                               
\end{lstlisting}
Um nun diese Berechnung f\"ur jedes Segment zu speichern, ben\"otigen
wir eine geeignete Datenstruktur. Dabei habe ich mich f\"ur ein Dictionary
entschieden. Mir ist bewusst, dass ein Array, wie es die Numpy-Bibliothek
bereitstellt, Laufzeiteffizienter w\"are. Ich m\"ochte jedoch einerseits
die vielen ungenutzten Elemente vermeiden und andererseits soll vermieden
werden unn\"otige Imports zu verwenden.\\
\indent In diesem Dictionary werden nun Paare $(a, b)$ auf andere Paare
$(p, m)$ abgebildet. Dabei geben $a$ und $b$ das Intervall an, $p$
ist der Preis des Segmentes und $m$ die Mitte, die zu jenem Preis gef\"uhrt
hat. Um wirklich alle Werte zu errechnen, wird die 
\texttt{segment\_price}-Funktion mit f\"ur alle $b\in[0, \texttt{len(inp)}]$
und alle dazugeh\"origen $a\in[0, b]$ aufgerufen. Dabei wird der Funktion
ein Slice der Input-Liste \"ubergeben, das alle Indizes $[a, b]$ umfasst.
\subsection{Segmentvereinigungen}
Auch hierf\"ur wird ein Dictionary verwendet. Dabei werden Paare $(i, b)$
auf $(p, a-1)$ abgebildet. Hierbei ist $i$ die Anzahl der Picks im Intervall
$[1, b]$, f\"ur das der Preis $p$ ermittelt wurde. Das Intervall $[a, b]$
ist im letzten Schritt hinzugekommen. Somit kann mit $a$ ermittelt werden,
wie die optimale Verteilung der Picks zusammengesetzt ist.\\
\indent F\"ur $i=1$ k\"onnen bereits alle Werte eingetragen werden, denn
diese sind \"aquivalent zu den Einzelsegmentpreisen. 
Um alle weiteren $(i, b)$ in das Dictionary einzutragen, werden zwei Schleifen
ben\"otigt. Einerseits $i$ im Intervall $[2, 10]$ (bzw. $[1, 9]$) und andererseits
$b$ im Intervall $[1, \texttt{len(inp)}]$. Nun wird man im Code das Intervall
$[i, \texttt{len(inp)}]$ finden. Das liegt daran, dass $i$ angibt, wie viele
Picks bereits vor dem Intervall lagen. Somit m\"ussen dort mindestens
genauso viele Werte liegen, ansonsten w\"are mindestens ein Pick doppelt und
die Verteilung somit suboptimal. F\"ur jedes $a$ wird dann der Gesamtpreis
aus dem Einzelsegment $(a, b)$ und dem bereits optimierten Segment $(1, a-1)$
berechnet. Sollte dieser Preis besser sein, als der bisher beste, so
wird er in das Dictionary eingetragen. Somit steht, nachdem alle $a$ betrachtet
wurden, die beste Aufteilung des Intervalls und der damit verbundene Preis
im Dictionary.\\
\indent Folglich beschreibt das Tupel $(10, \texttt{len(inp)}-1)$ am Ende den optimalen
Preis. Dieser Wert wird dann in \texttt{volldaneben\_solver.price} eingetragen.\\
\indent Zuletzt m\"ussen die tats\"achlichen Picks errechnet werden.
Hierzu wird f\"ur alle $i$ von 10 bis 1 die Mitte des Einzelsegments am
Ende des optimalen Intervalls berechnet. Dabei wird das Intervall dadurch
getrackt, dass $b$ zun\"achst $\texttt{len(inp)}-1$ ist und anschlie\ss end
Schritt mit Hilfe des in \texttt{combined\_prices} gespeicherten $a$
verkleinert wird. ($a-1$ ist das neue $b$)
\section{Beispiele}
Die Dateien f\"ur die Beispiele befinden sich im Ordner \texttt{beispieldaten}.
Da das Programm deterministisch arbeitet, sollten sich die hier dargestellten
Outputs reproduzieren lassen.\\
\indent Die ersten drei Beispiele entsprechen den Vorgaben von der
BwInf-Website. Anschlie\ss end habe ich mich mit den Primzahlen
unter 1000 besch\"aftigt.
\subsection{Beispiel 1 (BwInf)}
Datei: \texttt{beispieldaten/beispiel1.txt}\\
Ausgabe (\texttt{output/beispiel1.txt}):
\begin{verbatim}
Inputs:  [5, 10, 15, 20, 25, 30, 35, 40, 45, 50, 
          55, 60, 65, 70, 75, 80, 85, 90, 95, 100, 
          105, 110, 115, 120, 125, 130, 135, 140, 145, 150, 
          155, 160, 165, 170, 175, 180, 185, 190, 195, 200, 
          205, 210, 215, 220, 225, 230, 235, 240, 245, 250, 
          255, 260, 265, 270, 275, 280, 285, 290, 295, 300, 
          305, 310, 315, 320, 325, 330, 335, 340, 345, 350, 
          355, 360, 365, 370, 375, 380, 385, 390, 395, 400, 
          405, 410, 415, 420, 425, 430, 435, 440, 445, 450, 
          455, 460, 465, 470, 475, 480, 485, 490, 495, 500, 
          505, 510, 515, 520, 525, 530, 535, 540, 545, 550, 
          555, 560, 565, 570, 575, 580, 585, 590, 595, 600, 
          605, 610, 615, 620, 625, 630, 635, 640, 645, 650, 
          655, 660, 665, 670, 675, 680, 685, 690, 695, 700, 
          705, 710, 715, 720, 725, 730, 735, 740, 745, 750, 
          755, 760, 765, 770, 775, 780, 785, 790, 795, 800, 
          805, 810, 815, 820, 825, 830, 835, 840, 845, 850, 
          855, 860, 865, 870, 875, 880, 885, 890, 895, 900, 
          905, 910, 915, 920, 925, 930, 935, 940, 945, 950, 
          955, 960, 965, 970, 975, 980, 985, 990, 995]
Price: 4950
Picks: [945, 840, 735, 630, 525, 430, 335, 240, 145, 50]
\end{verbatim}
Man sieht, dass die Gleichverteilung der Eingaben f\"ur einen sehr
hohen Preis sorgt.
\subsection{Beispiel 2 (BwInf)}
Datei: \texttt{beispieldaten/beispiel2.txt}\\
Ausgabe (\texttt{ouptut/beispiel2.txt}):
\begin{verbatim}
Inputs:  [11, 12, 22, 27, 42, 43, 58, 59, 60, 67, 
          69, 84, 87, 91, 94, 123, 124, 135, 167, 170, 
          172, 178, 198, 211, 226, 229, 276, 281, 305, 313, 
          315, 324, 327, 335, 336, 362, 364, 367, 368, 368,
          370, 373, 383, 386, 393, 399, 403, 413, 421, 421, 
          426, 429, 434, 456, 492, 505, 526, 530, 537, 539, 
          540, 545, 567, 582, 584, 586, 649, 651, 676, 690, 
          729, 736, 739, 750, 754, 763, 777, 782, 784, 788, 
          793, 802, 808, 814, 846, 857, 862, 862, 873, 886, 
          895, 915, 919, 925, 926, 929, 932, 956, 980, 996]
Price: 1924
Picks: [929, 862, 777, 651, 539, 421, 368, 315, 172, 59]
\end{verbatim}
\subsection{Beispiel 3 (BwInf)}
Datei: \texttt{beispieldaten/beispiel3.txt}\\
Ausgabe (\texttt{output/beispiel3.txt}):
\begin{verbatim}
Inputs:  [20, 40, 60, 80, 100, 100, 120, 120, 140, 160, 
          160, 180, 200, 220, 220, 240, 240, 240, 240, 260, 
          260, 260, 280, 280, 300, 300, 340, 340, 340, 340, 
          360, 360, 380, 380, 400, 400, 420, 420, 440, 440, 
          460, 460, 460, 480, 480, 500, 520, 520, 520, 520, 
          520, 520, 540, 540, 540, 560, 580, 580, 580, 580, 
          600, 600, 620, 640, 640, 640, 660, 680, 680, 680, 
          680, 700, 700, 720, 720, 720, 720, 720, 740, 740, 
          760, 780, 780, 800, 800, 820, 840, 840, 860, 860, 
          860, 880, 900, 900, 920, 920, 960, 980, 980, 1000]
Price: 2160
Picks: [960, 860, 720, 660, 580, 520, 440, 340, 240, 100]
\end{verbatim}
\subsection{Beispiel 4 (Zweiundvierzig)}
Primzahlen $< 1000$\\
Datei: \texttt{beispieldaten/beispiel4.txt}\\
Ausgabe (\texttt{output/beispiel4.txt}):
\begin{verbatim}
Inputs:  [2, 3, 5, 7, 11, 13, 17, 19, 23, 29, 31, 
          37, 41, 43, 47, 53, 59, 61, 67, 71, 73, 
          79, 83, 89, 97, 101, 103, 107, 109, 113, 127, 
          131, 137, 139, 149, 151, 157, 163, 167, 173, 179, 
          181, 191, 193, 197, 199, 211, 223, 227, 229, 233, 
          239, 241, 251, 257, 263, 269, 271, 277, 281, 283, 
          293, 307, 311, 313, 317, 331, 337, 347, 349, 353, 
          359, 367, 373, 379, 383, 389, 397, 401, 409, 419, 
          421, 431, 433, 439, 443, 449, 457, 461, 463, 467, 
          479, 487, 491, 499, 503, 509, 521, 523, 541, 547, 
          557, 563, 569, 571, 577, 587, 593, 599, 601, 607, 
          613, 617, 619, 631, 641, 643, 647, 653, 659, 661, 
          673, 677, 683, 691, 701, 709, 719, 727, 733, 739, 
          743, 751, 757, 761, 769, 773, 787, 797, 809, 811, 
          821, 823, 827, 829, 839, 853, 857, 859, 863, 877, 
          881, 883, 887, 907, 911, 919, 929, 937, 941, 947, 
          953, 967, 971, 977, 983, 991, 997]
Price: 4055
Picks: [947, 853, 751, 653, 569, 457, 353, 239, 137, 37]
\end{verbatim}
\section{Quellcode}
\subsection{Hauptklasse}
\begin{lstlisting}[frame=single]
# The main class to solve the problem; just for some parameters to be pseudo global
class volldaneben_solver:

  # constructor ...
  def __init__(self, inputs):
    # sorting inputs for the algorithm to work
    self.inputs = sorted(inputs)
    # getting prices for intervals [a, b] with exactly one pick
    self.set_segment_prices()
    # getting prices for intervals [1, b] with two to ten picks
    self.set_combined_prices()
    # setting self.values to the actual values
    self.set_values()

  # see self.__init__
  def set_segment_prices(self): #[...]

  # calculates the minimal cost of a given segment with one pick
  def segment_price(self, segment): #[...]

  # see self.__init__
  def set_combined_prices(self): #[...]

  # setting the picks from the self.combined_prices
  def set_values(self): #[...]
  
\end{lstlisting}
\subsection{Einzelsegmentpreise}
\begin{lstlisting}[frame=single]
def set_segment_prices(self):
  # just shorter code
  inp = self.inputs
  # just shorter code
  l = len(inp)
  # (start, end) --> (price, pick)
  self.segment_prices = {}
  # cycling the end of the interval in [0, l)
  for b in range(l):
    # cycling the beginning of the interval in [0, b); a <= b!
    for a in range(b+1):
      # getting the price and writing it into the dictionary
      self.segment_prices[(a, b)] = (
        self.segment_price(inp[a:b+1]), (a+b)//2)                       

# calculates the minimal cost of a given segment with one pick
def segment_price(self, segment):
  # optimization & readability
  pick = segment[len(segment) // 2]
  # the price is the sum of deltas of each element of the segment to the pick
  return sum([abs(x - pick) for x in segment])

 
\end{lstlisting}
\subsection{Segmentvereinigung}
\begin{lstlisting}[frame=single]
def set_combined_prices(self):
    # readability
    inp = self.inputs
    l = len(inp)
    # (pick count, end) --> (price, previous end)
    self.combined_prices = {}
    # setting the prices of segments with one pick
    for b in range(l):
        self.combined_prices[(1, b)] = (self.segment_prices[(0, b)][0], 0)
    # [1, 9] (i) ^= [2, 10] (i') picks
    for i in range(1, 10):
        # the end can't be less than i, 'cause there are i picks already
        for b in range(i, l):
            # the beginning of the single pick interval at the end of the combined one
            for a in range(i-1, b):
                # combinded prices without the single pick segment [a, b]
                prev = self.combined_prices.get((i, a-1), (0, a))
                # total price is the previous price + the single segment price
                price = prev[0] + self.segment_prices[(a, b)][0]
                # there might have been a previous calculation; the current best
                best = self.combined_prices.get((i+1, b), (sys.maxsize, 0))
                # the new price is only relevant if it's actually lower than the old one
                if best[0] > price:
                    # if so, updated the current best for [1, b] with i' picks
                    self.combined_prices[(i+1, b)] = (price, a)
    # the acutal price is the very last one
    self.price = self.combined_prices[(10, l-1)]
\end{lstlisting}
\subsection{Wertberechnung}
\begin{lstlisting}[frame=single]
def set_values(self):
  # a list of picks ... (indices)
  picks = []
  # track [1, b]
  b = len(self.inputs)-1
  # each consecutive combined segment has one less pick
  for i in range(10, 0, -1):
    # get the data for the current segment
    current = self.combined_prices[(i, b)]
    # append the pick of the intervall [new b + 1, b] to picks
    picks.append(self.segment_prices[(current[1], b)][1])
    # the new end of the remaining interval
    b = current[1]-1
  # indices --> values
 self.picks = [self.inputs[x] for x in picks] 
\end{lstlisting}
\subsection{Hauptmethode} 
\begin{lstlisting}[frame=single]
# main method; solve the problem and generate outputs
def volldaneben():
    # if no path is given, read input
    if len(sys.argv) == 1:
        inputs = from_input()
    # if a path is given, read file
    else:                                                                           
        inputs = from_file(sys.argv[1])
    # solve the problem
    solver = volldaneben_solver(inputs)
    print('Inputs: ', solver.inputs)
    # price output
    print('Price:', solver.price[0])
    # picks output
    print('Picks:', solver.picks)
    # validating price from inputs and picks
    actual_price = get_price(solver.inputs, solver.picks)
    if solver.price[0] != actual_price:
        print('WARNING: Invalid output!')
        print('Actual price:', actual_price)

# calculating the price according to the task from inputs and picks
def get_price(inputs, picks):
    # price: sum of minimal differences of each input to any pick
    return sum([min([abs(i-p) for p in picks]) for i in inputs])

# calling the main function ...
if __name__ == '__main__':
    volldaneben()
\end{lstlisting}
\subsection{Einlesen}
\begin{lstlisting}[frame=single]
# I/O: reading inputs from file
def from_file(path):
  inputs = []
  with open(path) as f:
    l = f.readline().strip('\n')
    while l != '':
      inputs.append(int(l))
      l = f.readline().strip('\n')
  return inputs

# I/O: reading inputs from input stream; < - opeartor
def from_input():
  inputs = []
  l = input()
  while l != '':
    inputs.append(int(l))
    l = input()
  return inputs
\end{lstlisting}
\end{document}
