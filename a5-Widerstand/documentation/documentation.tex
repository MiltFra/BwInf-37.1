\documentclass[a4paper,10pt,ngerman]{scrartcl}
\usepackage{babel}
\usepackage[T1]{fontenc}
\usepackage[utf8x]{inputenc}
\usepackage[a4paper,margin=2.5cm,footskip=0.5cm]{geometry}

% Die nächsten drei Felder bitte anpassen:
\newcommand{\Aufgabe}{Aufgabe 5: Widerstand} % Aufgabennummer und Aufgabennamen angeben
\newcommand{\TeamID}{00922}       % Team-ID aus dem PMS angeben
\newcommand{\TeamName}{Zweiundvierzig} % Team-Namen angeben
\newcommand{\Namen}{Franz Miltz} % Namen der Bearbeiter/-innen dieser Aufgabe angeben
 
% Kopf- und Fußzeilen
\usepackage{scrlayer-scrpage, lastpage}
\setkomafont{pageheadfoot}{\large\textrm}
\lohead{\Aufgabe}
\rohead{Team-ID: \TeamID}
\cfoot*{\thepage{}/\pageref{LastPage}}

% Position des Titels
\usepackage{titling}
\setlength{\droptitle}{-1.0cm}

% Für mathematische Befehle und Symbole
\usepackage{amsmath}
\usepackage{amssymb}

% Für Bilder
\usepackage{graphicx}
\graphicspath{./images/}

% Für Algorithmen
\usepackage{algpseudocode}

% Für Quelltext
\usepackage{listings}
\usepackage{color}
\definecolor{mygreen}{rgb}{0,0.6,0}
\definecolor{mygray}{rgb}{0.5,0.5,0.5}
\definecolor{mymauve}{rgb}{0.58,0,0.82}
\lstset{
  keywordstyle=\color{blue},commentstyle=\color{mygreen},
  stringstyle=\color{mymauve},rulecolor=\color{black},
  basicstyle=\footnotesize\ttfamily,numberstyle=\tiny\color{mygray},
  captionpos=b, % sets the caption-position to bottom
  keepspaces=true, % keeps spaces in text
  numbers=left, numbersep=5pt, showspaces=false,showstringspaces=true,
  showtabs=false, stepnumber=2, tabsize=2, title=\lstname
}
\lstdefinelanguage{JavaScript}{ % JavaScript ist als einzige Sprache noch nicht vordefiniert
  keywords={break, case, catch, continue, debugger, default, delete, do, else, finally, for, function, if, in, instanceof, new, return, switch, this, throw, try, typeof, var, void, while, with},
  morecomment=[l]{//},
  morecomment=[s]{/*}{*/},
  morestring=[b]',
  morestring=[b]",
  sensitive=true
}

% Diese beiden Pakete müssen zuletzt geladen werden
%\usepackage{hyperref} % Anklickbare Links im Dokument
\usepackage{cleveref}

% Daten für die Titelseite
\title{\textbf{\Huge\Aufgabe}}
\author{\LARGE Team-ID: \LARGE \TeamID \\\\
	    \LARGE Team-Name: \LARGE \TeamName \\\\
	    \LARGE Bearbeiter/-innen dieser Aufgabe: \\ 
	    \LARGE \Namen\\\\}
\date{\LARGE\today}

\begin{document}

\maketitle
\tableofcontents

\vspace{0.5cm}

\section{Lösungsidee}
Das Problem l\"asst sich in zwei gleicherma\ss en anspruchsvolle Teile gliedern.
Einerseits muss herausgefunden werden, welche Widerst\"ande man wie zusammen bauen
kann und andererseits muss das Ergebnis visualisiert werden.
\subsection{Ermittlung der Widerst\"ande}
Um meine Probleml\"osung zu verstehen, muss man sich \"uberlegen, welche Arten von
Widerst\"anden es gibt. Gem\"a\ss\ der Aufgabenstellung k\"onnen Widerst\"ande entweder
elementar sein oder aus genau zwei Teilwiderst\"anden bestehen. Au\ss erdem wurde
durch die Limitierung auf vier Komponenten sichergestellt, dass die Stromrichtung stets 
eindeutig ist und jeder Widerstand in der Rechnung nur genau ein mal auftauchen kann.
(Danke daf\"ur an das BwInf-Team!)\\
\indent Das bedeutet, dass ein Widerstand mit vier Teilen immer aus zwei Widerst\"anden
besteht, die entweder beide zwei Elementarwiderst\"ande enthalten oder ein mal aus
einem und ein mal aus drei Komponenten bestehen. Die Menge der Elementarwiderst\"ande
sei $E$ und die Menge zusammengesetzter Widerst\"ande $C$. Dabei ist $C_i$ die Menge
aller Widerst\"ande, die sich aus $i$ der gegebenen Elementarwiderst\"ande bauen l\"asst.
\begin{align}
  C_1 &= E\\
  |C_i| &= 2 \cdot \sum_{k=1}^{int(\frac{i}{2})} |C_{k}| \cdot |C_{i-k}| 
\end{align}
Es k\"onnen also alle zusammengesetzten Widerst\"ande ermittelt 
werden, ohne die jeweils
gr\"o\ss eren zu kennen. Folglich kann damit begonnen werden alle 
Widerst\"ande aus zwei
Teilen zu generieren, dann jene aus drei Teilen usw. Dabei muss 
es jedoch vermieden werden,
dass Elementarwiderst\"ande mehrfach verwendet werden.
\subsection{Visualisierung}
Die Visualisierung ist von der Idee her ziemlich simpel. Das Ergebnis ist zwar nicht
optimal, erf\"ullt aber den Zweck. Durch die verschiedenen Widerstandsarten kann
hier rekursiv gearbeitet werden. Jeder Widerstand hat eine eigene Funktion, die
auf einem Bild in einem bestimmten Bereich jenen Widerstand einzeichnet. Dabei
verbinden Elementarwiderst\"ande die beiden Anschl\"usse und zeichnen einen Widerstand
in die Mitte. Sequentielle Widerst\"ande halbieren den Bereich horizontal, sodass
in die beiden Teilrechtecke die Teile gezeichnet werden k\"onnen. Prallele Widerst\"ande
haben einen Rand, in dem die Anschl\"usse mit denen der Teile verbunden werden.
Der restliche Teil des zur Verf\"ugung stehenden Bereiches wird dann vertikal geteilt,
sodass die beiden Teile \"ubereinander gezeichnet werden k\"onnen. In die Widerst\"ande
werden dann noch die Einzelwiderst\"ande gezeichnet und in die Ecke des Bildes wird
der Gesamtwiderstand geschrieben.\\
\indent Mit dieser Herangehensweise m\"ussen einige Konstanten von Anfang an gut
gew\"ahlt werden, dazu jedoch mehr in der Umsetzung.
\section{Umsetzung}
Das Programm wurde f\"ur Python 3.7.1 auf einem Linux-System geschrieben 
und nur dort
getestet. Ben\"otigt wird au\ss erdem eine Fork der PIL
(Python Imaging Library), Pillow.
\begin{equation*}
    \texttt{\$ pip install pillow} 
\end{equation*}
Des Weiteren wurde die GNU FreeMono Schriftart 
(https://www.fontspace.com/gnu-freefont/freemono)
verwendet. Das Programm (\texttt{widerstand.py}) sollte aus dem Ordner heraus
ausgef\"uhrt werden, damit auf die Schriftart zugegriffen werden kann.
Der Output befindet sich dann einerseits in der Konsole und andererseits wird
eine Bilddatei (\texttt{output/rXXX.png}) erstellt. Dabei ist im Dateinamen der
Gesamtwiderstand der Schaltung als ganze Zahl vorhanden. Die Software sollte
als Proof-of-Concept betrachtet werden, da es an Nutzerfreundlichkeit mangelt.\\
\indent Auch hier werde ich die Visualisierung getrennt vom Rest des Programms 
thematisieren. 
\subsection{Modellierung}
Das Problem umfasst drei verschieden Widerst\"ande. Diese wuden durch Klassen dargestellt.
Die Basis-Klasse, \texttt{Resistor}, hat einen Wert (\texttt{Resistor.value}) und kann diesen
\"uber \texttt{Reistor.get\_value} zur\"uckgeben. Au\ss erdem entspricht sowohl
die Struktur, als auch die String-Repr\"asentation einfach dem eigenen Wert.\\
\indent Weiterhin gibt es den \texttt{SequentialResistor}, der aus zwei Teilwiderst\"anden
(\texttt{part1} und \texttt{part2}) besteht, die in Reihe geschalten sind. Somit ist
der Gesamtwiderstand gleich der Summe der beiden Teilwiderst\"ande.
\begin{align}
    R = R_1 + R_2 
\end{align}
Der String besteht aus dem Gesamtwiderstand und der Struktur.
Die Struktur besteht aus der Art des Widerstandes (\texttt{seq}) sowie den
Strukturen der beiden Teilwiderst\"ande.\\ 
\indent Zuletzt existiert die Klasse (\texttt{ParallelResistor}), die
Teilwiderst\"ande in einer Parallelschaltung beschreibt. Folglich gilt f\"ur den
Gesamtwiderstand $R$ (\texttt{ParallelResistor.value}):
\begin{align}
  \frac{1}{R} &= \frac{1}{R_1} + \frac{1}{R_2} \\
  \Leftrightarrow \frac{1}{R} &= \frac{R_1 + R_2}{R_1 \cdot R_2}\\
  \Leftrightarrow R &=\frac{R_1\cdot R_2}{R_1 + R_2}
\end{align}
Die String-Repr\"asentation sowie die Struktur verhalten sich analog zum
sequentiellen Widerstand. Beide zusammegesetzte Widerst\"ande erfordern zwei
Teilwiderst\"ande im Konstruktor. Der Basis-Widerstand erfordert lediglich
einen Wert. 
\subsection{Berechnung der Widerst\"ande}
Mit der oben beschriebenen L\"osungsidee und der soeben thematisierten
Visualisierung wird es einfach, alle Widerst\"ande zu berechnen. Zun\"achst
muss jedoch die Liste der Elementarwiderst\"ande in einen \texttt{Counter},
d.h. ein Dictionary mit der Anzahl der Widerst\"ande, konvertiert werden.
Diese Elementarwiderst\"ande sind dann gleichzeitig zusammengesetzte
Widerst\"ande mit genau einem Teil.\\
\indent Wenn nun also die \texttt{set\_combined\_parts}-Funktion aufgerufen
wird, m\"ussen ein paar F\"alle behandelt werden. Sollte das \"ubergebene
$k<2$ sein, so wird wurde bereits alles erledigt. Wenn die Widerst\"ande
f\"ur $k-1$ noch nicht berechnet wurden, muss dies zun\"achst erledigt werden.
D.h., man muss die Methode nicht mehrmals aufrufen, die Rekursion sorgt
daf\"ur, dass alles n\"otige im Voraus berechnet wird.\\
\indent Nun m\"ussen alle M\"oglichkeiten, $k$ auf die beiden Teilwiderst\"ande
zu verteilen, durchlaufen werden. Au\ss erhalb dieser Schleife wird bereits
eine Kandidatenliste erstellt, die nach und nach gef\"ullt werden soll.
In der Schleife gibt es eine weitere Kandidatenliste, die am Ende auf
Duplikate \"uberpr\"uft wird, bevor sie zu den eigentlichen Kandidaten
hinzugef\"ugt wird. F\"ur jede der Verteilungen $(l, k-l)$ werden alle
zusammengesetzten Widerst\"ande mit $l$ Teilen mit allen zusammengesetzten
Widerst\"anden mit $(k-l)$ teilen kombiniert. F\"ur jede dieser Kombinationen
gibt es zwei Kandidaten, d.h. sequentiell und parallel.\\
\indent Das Entfernen der Duplikate erfolgt in einer sortierten Liste
(nach Gesamtwiderst\"anden), wobei benachbarte Elemente \"uberpr\"uft
werden. Hierbei ist ein Duplikat ein Widerstand, der den gleichen
Gesamtwiderstand mit den gleichen Elementarwiderst\"anden erzeugt.
Ich bin mir der Tatsache bewusst, dass dieser Ansatz nicht zu 100 Prozent 
funktioniert, da Duplikate eventuell durch andere Widerst\"ande mit
dem gleichen Gesamtwiderstand voneinander getrennt werden. In den meisten
F\"allen sollte diese Vorgehensweise jedeoch zielf\"uhrend sein. Eine
genauere Methode w\"are in Bezug auf den Rechenaufwand unproportional
zu dem Gewinn, der durch die verringerte Anzahl an Widerst\"anden
erzielt wird.
\subsection{Finden des besten Widerstandes}
Hierzu wird eine Bin\"arsuche verwendet, die eine wesentlich bessere
Laufzeit ($O(log(n))$) hat, als das einfache durchlaufen der
Widerstandsliste ($O(n)$). Leider wird dieser Vorteil dadurch relativiert,
dass die Liste sortiert werden muss, was bei dem Betrachten jedes
Elementes nicht der Fall w\"are.\\
\indent In der Aufgabenstellung wird gefordert, dass f\"ur jedes
$k \leq 4$ der zusammengesetzte Widerstand mit m\"oglichst geringer
Abweichung zu finden ist. Also wird eine Liste mit den besten Widerst\"anden
f\"ur jedes $k$ erstellt. Des Weiteren kommen in jedem Schritt nur Widerst\"ande
hinzu. Es kann also mit einer leeren Liste an Kandidaten begonnen werden und
in jeder Iteration wird die Liste erweitert und neu sortiert.
Es g\"abe sicherlich eine algorithmisch ausgepfeiltere Methode die neuen
Widerst\"ande mit den bereits vorhandenen zu vereinigen und gleichzeitig
zu sortieren (s. Merge-Sort), aber die Laufzeit ist nicht unannehmbar und
wie bereits erw\"ahnt handelt es sich um ein Proof-of-Concept und kein
nutzerfreundliches Endprodukt.\\
\indent Es folgt die Bin\"arsuche. Hierf\"ur wird ein Intervall solange
verkleinert, bis es entweder minimal gro\ss\ ist oder die Mitte genau
den gesuchten Wert enth\"alt. Dabei umfasst das Intervall $[a, b]$ zun\"achst
die gesamte Liste. Anschlie\ss end wird die Mitte $c=\frac{a+b}{2}$ berechnet.
Nun gibt es drei F\"alle (berachtet wird immer der Gesamtwiderstand des
Elementes):
\begin{enumerate}
  \item Die Mitte ist gr\"o\ss er als der gesuchte Wert.\\
        \Rightarrow Die Mitte ist das neue rechte Intervallende $b$.
  \item Die Mitte ist kleiner als der gesuchte Wert. \\
        \Rightarrow Die Mitte ist das neue linke Intervallende $a$.
  \item Die Mitte ist genauso gro\ss\ wie der gesuchte Wert.\\
        \Rightarrow Die Suche ist fertig.
\end{enumerate}
Nachdem die Suche beendet ist, muss nat\"urlich \"uberpr\"uft werden, welche
Abbruchbedingung erf\"ullt wurde. Entweder ist $a+1=b$, dann muss \"uberpr\"uft
werden, ob $a$ oder $b$ n\"aher am gesuchten Wert liegt oder das ist nicht so,
dann steht fest, dass $c$ den gesuchten Wert enth\"alt. Nun wird der beste
Wert der Liste f\"ur die besten Widerst\"ande f\"ur jedes $k$ hinzugef\"ugt. 
Sollte der exakte Wert gefunden worden sein, so wird die Liste mit jenem
Widerstand gef\"ullt, da der Widerstand von $k$ auch f\"ur $k+1$ verwendet
werden kann und keine geringere Abweichung mehr existiert. 
\subsection{Visualisierung}
Der Code f\"ur die Visualisierung befindet sich haupts\"achlich in den
Widerstandsklassen. Es gibt jedoch eine Funktion, die die erste 
\texttt{draw}-Funktion aufruft und ein Bild erstellt. Au\ss erdem ist diese
Funktion f\"ur das Abspeichern verantwortlich. Da es sich hierbei um eine
Bibliothek handelt, m\"ochte ich den Quellcode genauer thematisieren.
\lstset{language=Python}
\begin{lstlisting}
def draw_resistor(r, path):
  width, height = 800, 600
  im = Image.new('RGBA', (width, height), (255, 255, 255, 255))
  dctx = ImageDraw.Draw(im)
  fnt = ImageFont.truetype('font/FreeMono.ttf')
  r.draw(dctx, fnt, 0, 0, width, height,
        width//9, width//27)
  dctx.text((10, 10), 'R = ' + str(r.get_value()) + ' Ohm', fill=(0, 0, 0, 255))
  im.save(path+'.png')
\end{lstlisting}
Zun\"achst wird die Gr\"o\ss e des Bildes auf $800\times 600$ festgelegt. Diese
Werte haben keinen tieferen Sinn und eignen sich lediglich gut, um den Text lesbar
zu halten, w\"ahrend die Schaltung noch auf das Bild passt.\\
\indent Anschlie\ss end wird ein neues Bild erstellt, welches zun\"achst wei\ss 
gef\"ullt ist. Au\ss erdem wird die oben bereits beschriebene Schriftart geladen
und eine Draw-Umgebung (\emph{draw context}) erstellt. Nun kann der Widerstand mit Hilfe der
\texttt{draw}-Methode des Widerstandsobjektes auf das Bild gemalt werden.
\"Ubergeben wird der Bereich, in dem gemalt werden darf (das gesamte Bild),
die Schriftart, die Draw-Umgebung sowie die Gr\"o\ss e des Widerstandes als
Schaltungsbaustein. Diese Gr\"o\ss e ist wichtig. Zun\"achst habe ich 
geschaut, welches Seitenverh\"altnis gut aussieht und bin auf 3:1 gekommen. 
Nun ist es wichtig, dass gen\"ugend Widerst\"ande nebeneinander passen.
Hierbei k\"onnte man meinen, es m\"ussten nur vier sein. Da das Bild jedoch immer
halbiert wird und die Widerst\"ande nicht zwingend gleichm\"a\ss ig verteilt sind,
m\"ussen acht Bausteine einkalkuliert werden (In jeder H\"alfte bis zu vier).
Dabei kann der konstante Rand der parallelen Widerst\"ande unbeachtet bleiben,
da dieser nicht auftritt, wenn vier Widerst\"ande (die maximale Anzahl) in
Reihe geschalten werden. Folglich darf ein Widerstand nicht breiter als ein
Achtel des Bildes sein. Wenn man ein Neuntel w\"ahlt, sind die Verbindungen
zwischen den Widerst\"anden noch erkennbar. Solange das Bild nicht mehr als
drei mal so breit wie hoch ist, spielt die H\"ohe der Einzelwiderst\"ande
keine Rolle.\\
\indent Nachdem die Schaltung gemalt wurde, wird der Gesamtwiderstand in die
Ecke links oben geschrieben. Anschlie\ss end kann das Bild gespeichert werden.\\
\indent Der Code f\"ur das Zeichnen der Einzelwiderst\"ande ist sehr
unverst\"andlich. Am einfachsten ist hierbei der Reihenwiderstand, bei dem
nur die \texttt{draw}-Methoden der beiden Teile f\"ur die jeweiligen H\"alften
des vorgesehenen Bereiches aufgerufen werden. Auch relativ einfach ist der
Elementarwiderstand. Hier wird die Mitte berchnet und ein Rechteck 
gezeichnet, was die Gr\"o\ss e hat, die durch $\texttt{xr, yr}$ festgelegt
wurde. In das Rechteck wird zentriert Text geschrieben. Au\ss erdem werden
die Mitten der Bildr\"ander mit den Mitten der Rechtecksseiten verbunden.\\
\indent Wirklich un\"ubersichtlich ist die Parallelschaltung. Hier werden
zun\"achst acht Punkte, vier f\"ur jede Seite bestimmt. Diese Pukte bezeichnen
die vertikale Mitte des Bildrandes, die vertikale Mitte des inneren Randes
sowie die vertikalen Mitten der R\"ande der Teilwiderst\"ande. Diese werden
so, wie in den Beispielen zu sehen, verbunden. Falls es sie genauer interessiert,
werfen Sie bitte einen Blick in den Quellcode und zeichnen Sie es ich auf, denn
das ist zielf\"uhrender als es hier in Worte zu fassen.
Nachdem der ben\"otigte Teil der Schaltung gemalt wurde, k\"onnen nun die
Teilwiderst\"ande in den vorgesehenen Bereichen dargestellt werden. 
\section{Beispiele}
Die Beispiele sind auf Grund der Bilder ziemlich umfangreich.
Zu jedem Beispiel geh\"ort eine Konsolenausgabe
(die sich auch als \texttt{.txt}-Datei im \texttt{output}-Ordner finden l\"asst)
sowie vier Bilder. 
\newcommand{\images}[1]{
  Bild f\"ur $k=1$:
  \begin{center}
    \includegraphics[width=.8\textwidth]{images/r#1-k1.png}  
  \end{center}
  Bild f\"ur $k=2$:
  \begin{center}
    \includegraphics[width=.8\textwidth]{images/r#1-k2.png}  
  \end{center}
  \pagebreak
  Bild f\"ur $k=3$:
  \begin{center}
    \includegraphics[width=.8\textwidth]{images/r#1-k3.png}  
  \end{center}
  Bild f\"ur $k=4$:
  \begin{center}
    \includegraphics[width=.8\textwidth]{images/r#1-k4.png}  
  \end{center}
}
\pagebreak
\subsection{Beispiel 1 (R=500)}
Ausgabe:
\begin{verbatim}
k=1: 470.0
k=2:   500.3802: par(4700.0, 560.0)
k=3:   500.0000: seq(100.0, seq(180.0, 220.0))
k=4:   500.0000: seq(100.0, seq(180.0, 220.0))
\end{verbatim}
\images{500}
\pagebreak
\subsection{Beispiel 2 (R=140)}
\begin{verbatim}
k=1: 150.0
k=2:   140.4255: par(150.0, 2200.0)
k=3:   139.9494: par(2700.0, par(820.0, 180.0))
k=4:   140.0000: par(150.0, seq(680.0, seq(220.0, 1200.0)))
\end{verbatim}
\images{140}
\pagebreak
\subsection{Beispiel 3 (R=314)}
\begin{verbatim}
k=1: 330.0
k=2:   314.7265: par(6800.0, 330.0)
k=3:   314.0556: par(330.0, seq(1800.0, 4700.0))
k=4:   313.9993: par(1000.0, seq(120.0, par(1200.0, 470.0)))
\end{verbatim}
\images{314}
\pagebreak
\subsection{Beispiel 4 (R=315)}
\begin{verbatim}
k=1: 330.0
k=2:   314.7265: par(330.0, 6800.0)
k=3:   315.0000: par(560.0, seq(390.0, 330.0))
k=4:   315.0000: par(560.0, seq(390.0, 330.0))
\end{verbatim}
\images{315}
\pagebreak
\subsection{Beispiel 5 (R=1620)}
\begin{verbatim}
k=1: 1500.0
k=2:  1620.0000: seq(1500.0, 120.0)
k=3:  1620.0000: seq(1500.0, 120.0)
k=4:  1620.0000: seq(1500.0, 120.0)
\end{verbatim}
\images{1620}
\pagebreak
\subsection{Beispiel 6 (R=2719)}
\begin{verbatim}
k=1: 2700.0
k=2:  2700.0000: seq(1200.0, 1500.0)
k=3:  2720.0000: seq(1000.0, seq(220.0, 1500.0))
k=4:  2719.0000: seq(par(220.0, 180.0), seq(820.0, 1800.0))
\end{verbatim}
\images{2719}
\pagebreak
\subsection{Beispiel 7 (R=4242)}
\begin{verbatim}
k=1: 3900.0
k=2:  4230.0000: seq(3900.0, 330.0)
k=3:  4240.7767: seq(3900.0, par(390.0, 2700.0))
k=4:  4242.0000: seq(seq(270.0, 3900.0), par(120.0, 180.0))
\end{verbatim}
\images{4242}
\pagebreak
\subsection{Beispiel 8 (R=1337)}
\begin{verbatim}
k=1: 1200.0
k=2:  1330.0000: seq(330.0, 1000.0)
k=3:  1336.7961: seq(680.0, par(3300.0, 820.0))
k=4:  1336.9985: par(2200.0, seq(3300.0, par(390.0, 150.0)))
\end{verbatim}
\images{1337}
\pagebreak
\subsection{Beispiel 9 (R=2018)}
\begin{verbatim}
k=1: 2200.0
k=2:  2020.0000: seq(820.0, 1200.0)
k=3:  2018.1818: seq(1200.0, par(1800.0, 1500.0))
k=4:  2018.0091: seq(1500.0, par(820.0, par(3900.0, 2200.0)))
\end{verbatim}
\images{2018}
\pagebreak
\section{Quellcode}
Ich habe fast alle Kommentare entfernt und der Code folgt nicht mehr
dem pep8-Standard, da die vielen L\"ucken nur Platzverschwendung w\"aren.
\lstset{language=Python}
\subsection{Imports}
\begin{lstlisting}
from PIL import Image, ImageDraw, ImageFont
from collections import Counter
import sys
from pathlib import Path
\end{lstlisting}
\subsection{Widerstandsklassen}
\begin{lstlisting}[frame=single]
class Resistor():      
  def __init__(self, value):    
    self.value = value
    self.parts = [value, ]
  def __str__(self):
    return str(self.value)
  def get_structure(self):
    return str(self.value)
  def get_value(self):
    return self.value
  def draw(self, dctx, fnt, x1, y1, x2, y2, xr, yr):
    center = ((x1 + x2) // 2, (y1 + y2) // 2)
    dctx.line((x1, center[1], center[0]-xr//2, center[1]), fill=(0, 0, 0, 255))
    dctx.line((center[0]+xr//2, center[1], x2, center[1]), fill=(0, 0, 0, 255))
    dctx.rectangle((center[0]-xr//2, center[1]-yr//2, center[0] +
      xr//2, center[1]+yr//2), outline=(0, 0, 0, 255), fill=(0, 100, 200, 255))
    t = str(self.get_value()) + ' Ohm'
    ts = dctx.textsize(t)
    dctx.text((center[0]-ts[0]//2, center[1]-ts[1]//2), t, fill=(0, 0, 0, 255))

class ParallelResistor():
  def __init__(self, part1, part2):
    self.part1 = part1
    self.part2 = part2
    self.parts = self.part1.parts + self.part2.parts 
  def get_value(self):                                 
    return (self.part1.get_value()
            * self.part2.get_value()) \
        / (self.part1.get_value()
            + self.part2.get_value())                
  def draw(self, dctx, fnt, x1, y1, x2, y2, xr, yr):
    margin = xr // 3
    lc = (x1, (y1 + y2) // 2)         # center, left
    lm = (x1+margin, lc[1])           # center of margin, left
    llc = (lm[0], lm[1]+(y2-y1)//4)   # lower corner, left
    luc = (lm[0], lm[1]-(y2-y1)//4)   # upper corner, left
    rc = (x2, lc[1])                  # center, right
    rm = (x2-margin, rc[1])           # center of margin, right
    rlc = (rm[0], rc[1]+(y2-y1)//4)   # lower corner, right
    ruc = (rm[0], rc[1]-(y2-y1)//4)   # upper corner, right
    dctx.line((lc, lm), fill=(0, 0, 0, 255))
    dctx.line((rc, rm), fill=(0, 0, 0, 255))
    dctx.line((llc, luc), fill=(0, 0, 0, 255))
    dctx.line((rlc, ruc), fill=(0, 0, 0, 255))
    self.part1.draw(dctx, fnt, x1+margin,
                    y1, x2-margin, (y1+y2)//2, xr, yr)
    self.part2.draw(dctx, fnt, x1+margin,
                    (y1+y2)//2, x2-margin, y2, xr, yr)
  def get_structure(self):
    return f'par({self.part1.get_structure()}, {self.part2.get_structure()})'
  def __str__(self):
    return f'{self.get_value():>10.4f}: {self.get_structure()}'

class SequentialResistor():
  def __init__(self, part1, part2):
    self.set_parts(part1, part2)
    self.parts = self.part1.parts + self.part2.parts
  def set_parts(self, p, q):
    if p.get_value() > q.get_value():
      self.set_parts(q, p)
    self.part1 = p
    self.part2 = q
  def draw(self, dctx, fnt, x1, y1, x2, y2, xr, yr):
    self.part1.draw(dctx, fnt, x1, y1, (x1+x2)//2, y2, xr, yr)
    self.part2.draw(dctx, fnt, (x1+x2)//2, y1, x2, y2, xr, yr)
  def get_value(self):
    return self.part1.get_value() + self.part2.get_value()
  def get_structure(self):
    return f'seq({self.part1.get_structure()}, {self.part2.get_structure()})'
  def __str__(self):
    return f'{self.get_value():>10.4f}: {self.get_structure()}'
\end{lstlisting}
\subsection{Finder-Klasse}
\begin{lstlisting}[frame=single]
class ResistorFinder:
  def __init__(self, max_parts, parts, img_path=str(Path.home())):
    # constant to define how many resistors can be combined
    self.k = max_parts
    self.img_path = img_path
    # initiates the base parts
    self.set_base_parts(parts)
    # calculates all the combined parts
    self.set_combined_parts(self.k)
    self.combined_parts = [sorted(self.combined_parts[i], key=lambda x: x.get_value(
    )) for i in range(len(self.combined_parts))]
  def set_base_parts(self, parts): #[...]
  def get(self, value, max_k=4): #[...]
  def set_combined_parts(self, k): #[...]
  def remove_duplicates(self, resistors): #[...]
  def base_parts_available(self, resistor):
    c = Counter(resistor.parts)      
    for key in c.keys():             
        if c[key] > self.base_parts.get(key, 0):
            return False             
    return True                     
\end{lstlisting}
\subsection{Finden aller Widerst\"ande}
\begin{lstlisting}[frame=single]
def set_base_parts(self, parts):
  self.base_parts = Counter()
  self.combined_parts = [[], ]
  for p in parts:                                       
    self.base_parts[p] = self.base_parts.get(p, 0) + 1  
    self.combined_parts[0].append(Resistor(p))

def set_combined_parts(self, k):
  cb = self.combined_parts
  if k < 2: 
    return
  if len(cb) < k - 1:
    self.set_combined_parts(k - 1)
  print(f'Finding for k={k}')
  candidates = []
  count = 0
  max_count = sum([len(cb[i-1]) * len(cb[k-i-1]) for i in range(1, k//2+1)])
  for l in range(1, k // 2 + 1):
    sub_candidates = []
    for p1 in cb[l-1]:    
      for p2 in cb[k-l-1]:
        par = ParallelResistor(p1, p2)
        if self.base_parts_available(par):
          sub_candidates.append(par)
        seq = SequentialResistor(p1, p2)
        if self.base_parts_available(seq):
          sub_candidates.append(seq)
        count += 1
        print(f'{count/max_count*100:.1f} %', end='\r')
    self.remove_duplicates(sub_candidates)
    candidates.extend(sub_candidates)
  self.remove_duplicates(candidates)
  cb.append(candidates)
  print(f'Found all for k={k}')
\end{lstlisting}
\subsection{Finden und Zur\"uckgeben des besten Widerstandes}
\begin{lstlisting}[frame=single]
def get(self, value, max_k=4):
  best_by_k = []
  parts = []
  for i, p in enumerate(self.combined_parts):
    parts.extend(p)
    parts = sorted(parts, key=lambda x: x.get_value())
    a, b = 0, len(parts)-1
    while a + 1 != b:
      c = (a + b) // 2
      c_value = parts[c].get_value()
      if c_value > value:
        b = c
      elif c_value < value:
        a = c
      else:
        break
    if a + 1 == b:
      if abs(value-parts[a].get_value()) < abs(value-parts[b].get_value()):
        best_by_k.append(parts[a])
      else:
        best_by_k.append(parts[b])
    else:
      for j in range(i, len(self.combined_parts)):
        best_by_k.append(parts[c])
      break
  for i, r in enumerate(best_by_k):
    draw_resistor(r, f'output/r{value}-k{i+1}')
  with open(f'output/r{value}.txt', 'w') as f:
    for i, r in enumerate(best_by_k):
      f.write(f'k={i+1}: {r}\n')
  return best_by_k
\end{lstlisting}
\subsection{Hauptfunktion}
\begin{lstlisting}[frame=single]
def widerstand():
  path = 'bwinf37-runde1/a5-Widerstand/beispieldaten/test'
  if len(sys.argv) > 1:
    path = sys.argv[1]
  with open(path) as f:
    parts = [float(x) for x in f.readlines()]
  print(parts)
  rf = ResistorFinder(4, parts)
  while True:
    v = float(input('R: '))
    print('----------------')
    options = rf.get(v)
    for i, r in enumerate(options):
      print(f'k={i+1}: {r}')
    print('----------------')

if __name__ == '__main__':
    widerstand()
 
\end{lstlisting}
\end{document}
