\documentclass[a4paper,10pt,ngerman]{scrartcl}
\usepackage{babel}
\usepackage[T1]{fontenc}
\usepackage[utf8x]{inputenc}
\usepackage[a4paper,margin=2.5cm,footskip=0.5cm]{geometry}

% Die nächsten drei Felder bitte anpassen:
\newcommand{\Aufgabe}{Aufgabe 1: Superstar} % Aufgabennummer und Aufgabennamen angeben
\newcommand{\TeamID}{00922}       % Team-ID aus dem PMS angeben
\newcommand{\TeamName}{Zweiundvierzig} % Team-Namen angeben
\newcommand{\Namen}{Franz Miltz} % Namen der Bearbeiter/-innen dieser Aufgabe angeben
 
% Kopf- und Fußzeilen
\usepackage{scrlayer-scrpage, lastpage}
\setkomafont{pageheadfoot}{\large\textrm}
\lohead{\Aufgabe}
\rohead{Team-ID: \TeamID}
\cfoot*{\thepage{}/\pageref{LastPage}}

% Position des Titels
\usepackage{titling}
\setlength{\droptitle}{-1.0cm}

% Für mathematische Befehle und Symbole
\usepackage{amsmath}
\usepackage{amssymb}

% Für Bilder
\usepackage{graphicx}

% Für Algorithmen
\usepackage{algpseudocode}

% Für Quelltext
\usepackage{listings}
\lstset{language=Python}
\usepackage{color}
\definecolor{mygreen}{rgb}{0,0.6,0}
\definecolor{mygray}{rgb}{0.5,0.5,0.5}
\definecolor{mymauve}{rgb}{0.58,0,0.82}
\lstset{
  keywordstyle=\color{blue},commentstyle=\color{mygreen},
  stringstyle=\color{mymauve},rulecolor=\color{black},
  basicstyle=\footnotesize\ttfamily,numberstyle=\tiny\color{mygray},
  captionpos=b, % sets the caption-position to bottom
  keepspaces=true, % keeps spaces in text
  numbers=left, numbersep=5pt, showspaces=false,showstringspaces=true,
  showtabs=false, stepnumber=2, tabsize=2, title=\lstname
}
\lstdefinelanguage{JavaScript}{ % JavaScript ist als einzige Sprache noch nicht vordefiniert
  keywords={break, case, catch, continue, debugger, default, delete, do, else, finally, for, function, if, in, instanceof, new, return, switch, this, throw, try, typeof, var, void, while, with},
  morecomment=[l]{//},
  morecomment=[s]{/*}{*/},
  morestring=[b]',
  morestring=[b]",
  sensitive=true
}
% Diese beiden Pakete müssen zuletzt geladen werden
%\usepackage{hyperref} % Anklickbare Links im Dokument
\usepackage{cleveref}
\usepackage{csquotes}
% Daten für die Titelseite
\title{\textbf{\Huge\Aufgabe}}
\author{\LARGE Team-ID: \LARGE \TeamID \\\\
	    \LARGE Team-Name: \LARGE \TeamName \\\\
	    \LARGE Bearbeiter/-innen dieser Aufgabe: \\ 
	    \LARGE \Namen\\\\}
\date{\LARGE\today}

\begin{document}

\maketitle
\tableofcontents

\vspace{0.5cm}

\section{Lösungsidee}
Das Problem besteht darin einen Algorithmus zu finden, f\"ur den der Average Case bei stochastisch
gleichverteilten Eingaben optimal ist. In anderen Worten: W\"urde man alle m\"oglichen Anordnungen der
Eingaben gleichzeitig betrachten, so w\"are der Preis minimal.\\
\indent Um das Problem zu modellieren wird offensichtlich ein Graph ben\"otigt, bei dem die
Knoten die Personen/Accounts sind und die Kanten beschreiben, wer wem folgt. Ein Graph ist ein
Tupel aus einer endlichen Menge von Knoten und einer endlichen Menge Kanten. Eine Kante ist ein
geordnetes Paar von Knoten (der Graph ist offensichtlich gerichtet). In unserem Fall kann eine Kante
nur existieren oder eben nicht. Eine Markierung ist nicht notwendig. Au\ss erdem kann es maximal
von jedem Knoten zu jedem anderen Knoten genau eine Kante geben. 
\begin{align}
  V &= \{v_1, v_2, .., v_n\} \\
  |V| &=: N\\
  E &= \{(V_a, V_b), (V_c, V_d), ...\} \text{ mit } a,b,c,d \in [1, N] \subset \mathbb{N}\\
  |E| &\leq N\cdot(N-1)
\end{align}
Das Ziel ist es nun so schnell wie m\"oglich alle Knoten als Superstars
auszuschlie\ss en. Man muss also versuchen, aus jeder Anfrage so viel wie m\"oglich
Information zu erhalten. Die folgenden \"Uberlegungen sind essenziell:
\begin{enumerate}
  \item Wenn Person A Person B folgt, k\"onnen wir Person A als Superstar
        ausschlie\ss en.
  \item Wenn Person A Person B nicht folgt, k\"onnen wir Person B als Superstar
        ausschlie\ss en.
\end{enumerate}  
Wenn wir uns nun zwei Mengen, Kandidaten $K$ und Nicht-Kandidaten $\bar{K}$, zur Hilfe nehmen, gilt Folgendes:
\begin{align}
  K \cup \bar{K} &= V\\
  K \cap \bar{K} &= \emptyset\\
  (v_1, v_2)\in E &\Rightarrow v_1 \in \bar{K}\\
  (v_1, v_2)\notin E &\Rightarrow v_2 \in \bar{K}
\end{align}
Wenn man also ein Paar $(v_1, v_2)$ betrachtet, wobei beide Knoten noch potentielle
Superstars sind, kann man genau einen der beiden Knoten ausschlie\ss en, wenn man wei\ss,
ob solch eine Kante existiert. Somit sollten erst m\"oglichst viele Kandidaten \"uber diesen
Weg ausgeschlossen werden.\\
\indent Sobald nurnoch ein Kandidat \"ubrig ist, m\"ussen alle Beziehungen dieses einzelnen
Knotens mit allen anderen Knoten \"uberpr\"uft werden. Dabei ist es vollkommen egal, ob
zun\"achst gepr\"uft wird, ob alle Knoten diesem Knoten folgen, oder, ob diese Knoten keinem 
anderen Knoten folgt.\\
\indent Mit diesem Ansatz wird nun sichergestellt, dass die optimalen Anfragen unter
Ber\"ucksichtigung des zur Verf\"ugung stehenden Wissens gestellt werden.
\section{Umsetzung}
Die Umsetzung erfolgte in Python 3 und der einzige \texttt{import} war \texttt{sys}, um
auf die Attribute beim Ausf\"uhren in der Command-Line zuzugreifen. Die Sprache eignet
sich insofern gut, als dass der Code relativ kurz ist und bereits Mengen
(\texttt{set}) implementiert
wurden, was angesichts der L\"osungsidee sehr Vorteilhaft ist.\\
\indent Das Programm kann genutzt werden, indem es mit dem Python3-Interpreter und einer
Beispieldatei als Argument aufgerufen wird. Die Datei kann auf Linux-Systemen direkt
ausgef\"uhrt werden. Der Output erfolgt durch den Output-Stream, d.h. direkt in die 
Konsole.\\
\indent Es ist nat\"urlich unm\"oglich stets die optimale L\"osung zu finden, da es dem
Programm nicht erlaubt ist auf die daf\"ur notwendigen Informationen R\"ucksicht zu
nehmen. Auf Grund der verwendeten Mengen und deren Interation ist das Programm 
\textbf{nicht deterministisch}. Das wird sich vor allem in den Beispielen \"au\ss ern.
\subsection{Graph}
Zun\"achst wurde ein Graph implementiert. Dieser hat folgende Attribute:
\begin{itemize}
  \item \texttt{n}: \texttt{int}, Anzahl der Knoten 
  \item \texttt{v}: \texttt{set} mit Knoten als Elemente
  \item \texttt{e}: \texttt{set} mit Beziehungen \texttt{(v1, v2)} als Eintr\"age
  \item \texttt{price}: \texttt{int}, Anzahl der Anfragen bzw. Preis in Euro
\end{itemize} Hinzu kommen die Funktionen: 
\begin{itemize}
  \item \texttt{\_\_init\_\_(self, path)}: Konstruktor, setzt Attribute basierend auf Eingabedatei
  \item \texttt{request(self, node1, node2)}:\\ Gibt \texttt{True} zur\"uck, wenn es die Kante 
        \texttt{(node1, node2)} gibt; inkrementiert den Preis
\end{itemize}
Diese Implementation ist nahezu minimal, aber vollkommen ausreichend f\"ur das Problem.
Der eigentliche Algorithmus ben\"otigt nur diesen Graphen, um Anfragen zu stellen.
\subsection{Hauptteil}
Zur L\"osung des Problems, wurde eine Klasse mit folgenden Attributen erstellt:
\begin{itemize}
  \item \texttt{g}: Der Graph
  \item \texttt{links}: \texttt{set}, die bereits abgefragten Kanten
  \item \texttt{candidates}: \texttt{set}, zu Beginn enth\"alt es alle Elemente aus \texttt{g.v} ($K$)
  \item \texttt{not\_candidates}: \texttt{set}, zu Beginn leer ($\bar{K}$)
\end{itemize}
Auff\"allig ist hier, dass die Kanten auch hier als \texttt{set} verwaltet werden, obwohl
somit der tats\"achliche Wert dieser Kanten verloren geht. Das ist m\"oglich, weil unmittelbar
nach jeder Anfrage an den Graphen bereits entschieden werden kann, welche Konsequenzen das
Ergebnis hat, und dieser Wert sp\"ater \"uberfl\"ussig wird.\\
\indent Das Programm wird durch eine \texttt{run}-Funktion ausgef\"uhrt. Diese enth\"alt
eine Schleife, die immer wieder \texttt{request\_best} aufruft, bis entweder jener Aufruf
wahr zur\"uckgibt oder keine Kandidaten mehr \"ubrig sind.
\indent Weiterhin interessant ist die \texttt{request}-Funktion. Sie ist dazu da Anfragen
an den Graphen zu stellen und das Erebnis auszuwerten. Dabei wird zuerst die Anfrage gestellt
und wenn entweder \texttt{node1} ein Kandidat ist und jemandem folgt oder \texttt{node2}
ein Kandidat ist, dem von \texttt{node1} nicht gefolgt wird, wird der jeweilige Knoten
aus $K$ entfernt und zu $\bar{K}$ hinzugef\"ugt.\\
\indent Nun ist es die Aufgabe der \texttt{request\_best}-Funktion einen Funktionsaufruf 
von \texttt{request} durchzuf\"uhren, der m\"oglichst optimal ist. Dabei wird zun\"achst
eine M\"oglichkeit gesucht eine Anfrage zwischen zwei verschiedenen Kandidaten 
durchzuf\"uhren. Sollte das m\"oglich sein, so gibt die Funktion \texttt{False} zur\"uck. Wenn nicht,
wird mit dem zweiten Teil, der \texttt{check\_last}-Funktion fortgefahren. Hier wird f\"ur
alle Knoten in $\bar{K}$ gepr\"uft, ob eine zu dem letzten Kandidaten noch nicht gepr\"uft
wurde. Auch hier wird \texttt{False} zur\"uckgegeben, wenn solch eine Anfrage m\"oglich
ist. Sollte auch nach diesem Muster keine m\"ogliche Anfrage gefunden werden, kann wahr
zur\"uckgegeben werden, denn der letzte Knoten ist nun mit Sicherheit ein Superstar, denn
er wurde weder bei den Anfragen, ob der Knoten jemandem folgt, noch bei den Anfragen, ob
jemand dem Knoten folgt, aus der Menge der Kandidaten entfernt.
\subsection{Ergebnis}
Mit diesem Algorithmus liegt der optimale Fall bei $2(N-1)$ Anfragen/Euro, insofern 
ein Superstar exisitiert. Denn dann sind alle Anfragen, die Kandidaten ausschlie\ss en 
bereits Teil der unbedingt notwendigen Anfragen, die einen Superstar verifizieren. Sollte
kein Superstar exisitieren, liegt der Best-Case nat\"urlich bei $N$, da jede Kante genau
einen Knoten ausschlie\ss en kann. Beide F\"alle sind sehr unwahrscheinlich.\\
\indent Der Worst-Case mit und ohne Superstar ist $3(N-1)$, da zun\"achst alle Knoten bis
auf einen Kandidaten ausgeschlossen werden k\"onnen und im Anschluss alle Verfikationskanten
($=2(N-1)$) gerp\"uft werden. Die letzte Anfrage legt dann fest, ob der letzte Kandidat
ein Superstar ist oder nicht.
\section{Beispiele}
In diesem Programm wurde kein Benchmark durchgef\"uhrt, was die ben\"otigte Zeit
ber\"ucksichtigt, da jene vernachl\"assigbar gering ist.\\
\indent Weiterhin werde ich nur bei den Beispielen 1, 2 und 3 die Folge von Anfragen angeben, da
es ansosnten zu viele werden. Es ist klar, dass beim erneuten Ausf\"uhren mit hoher
Wahrscheinlichkeit ein anderer Preis zustande kommt. Somit sollten Sie sich nicht
wundern, wenn sie die Ergebnisse nicht bzw. nicht sofort reproduzieren k\"onnen. Leider
ist es au\ss erdem unm\"oglich f\"ur alle Eingaben alle M\"oglichkeiten auszuprobieren,
da die Zahl jener M\"oglichkeiten mit $N!$ berechnet wird und somit schneller als
exponentiell w\"achst.
\subsection{Beispiel 1 (BwInf)}
Befehl: \texttt{./superstar.py beispieldaten/superstar1.txt}\\
Ausgabe:
\begin{verbatim}
Selena -> Justin
Justin -> Hailey
Justin -> Selena
Hailey -> Justin

Total cost: EUR 4
Superstar found: Justin
\end{verbatim}
Die ersten f\"unf Zeilen sind die durchgef\"uhrten Anfragen. 
\textquote{\texttt{Selena -> Justin}}
entspricht also \textquote{\texttt{Folgt Selena Justin?}}.
In den letzten beiden Zeilen befinden sich der Preis f\"ur diese Untersuchung
sowie das Ergebnis. In diesem Fall wurden vier Anfragen gestellt, es kostet
also 4 Euro. Das Ergebnis ist, dass Justin der Superstar ist.
\subsection{Beispiel 2 (BwInf)}
Befehl: \texttt{./superstar.py beispieldaten/superstar2.txt}\\
Ausgabe:
\begin{verbatim}
Dijkstra -> Codd
Dijkstra -> Turing
Dijkstra -> Hoare
Dijkstra -> Knuth
Codd -> Dijkstra
Knuth -> Dijkstra
Hoare -> Dijkstra
Turing -> Dijkstra

Total cost: EUR 8
Superstar found: Dijkstra
\end{verbatim}
Diese Ausgabe scheint optimal zu sein, da alle Anfragen den Superstar beinhalten.
\subsection{Beispiel 3 (BwInf)}
Befehl: \texttt{./superstar.py beispieldaten/superstar3.txt}\\
Ausgabe:
\begin{verbatim}
Sjoukje -> Rinus
Sjoukje -> Edsger
Edsger -> Jitse
Jitse -> Rineke
Jitse -> Peter
Jitse -> Pia
Jitse -> Jorrit
Sjoukje -> Jitse

Total cost: EUR 8
There is no superstar in this graph!
\end{verbatim}
\subsection{Beispiel 4 (BwInf)}
Befehl: \texttt{./superstar.py beispieldaten/superstar4.txt}\\
Ausgabe:
\begin{verbatim}
Clara -> Oliver
Clara -> Sixten
...
Oscar -> Folke
Jack -> Folke

Total cost: EUR 160
Superstar found: Folke 
\end{verbatim}
Diese Ausgabe ist ebenfalls relativ gut, Preise von 220+ Euro sind bei diesem 
Graphen nicht abwegig.
\subsection{Beispiel 5 (Zweiundvierzig)}
Da das Programm nicht deterministisch arbeitet, wird es unm\"oglich einen Spezialfall
zu finden, bei dem entweder immer die optimale L\"osung gefunden wird, oder die Ausgabe
immer dem Worst-Case entspricht. Somit kann sich an dieser Stelle kein Beispiel befinden,
das eine solche Eigenheit das Programms aufzeigt.
\section{Quellcode}
S\"amtliche Inputs und Outputs sowie Kommentare wurden aus dem Code entfernt, k\"onnen
aber in \texttt{superstar.py} gefunden werden.
\subsection{Graph}
\begin{lstlisting}[frame=single]
class graph:
  def __init__(self, path):                           
      with open(path) as f:                           
          lines = f.readlines()                      
          self.v = set(lines[0].rstrip('\n').split(' ')) 
          self.n = len(self.v)                        
          self.e = set()                             
          for l in [x.rstrip('\n').split(' ') for x in lines[1:]]:
              if len(l) == 2:
                  self.e.add((l[0], l[1]))            
          self.price = 0                             
  def request(self, node1, node2):                  
      self.price += 1    
      return (node1, node2) in self.e    
\end{lstlisting}
\subsection{Star-Finder}
\begin{lstlisting}[frame=single]
class star_finder:                                      

    def __init__(self, g):                              
        self.g = g                                     
        self.links = set()                            
        self.candidates = g.v.copy()
        self.not_candidates = set() 

    def request(self, node1, node2):             
        r = self.g.request(node1, node2)        
        self.links.add((node1, node2))         
        if r and node1 in self.candidates:       
            self.candidates.remove(node1)        
            self.not_candidates.add(node1)      
        elif not r and node2 in self.candidates:      
            self.candidates.remove(node2)            
            self.not_candidates.add(node2)

    def run(self):
        while len(self.candidates) > 0:             
            if self.request_best():                
                break

    def request_best(self):  
        if len(self.candidates) > 1:              
            for i in self.candidates:            
                for j in self.candidates:       
                    if j != i:                 
                        self.request(i, j)    
                        return False         
        return self.check_last()            

    def check_last(self):
        for c in self.candidates:
            for n in self.not_candidates:  
                if n != c:                
                    if (n,c) not in self.links:
                        self.request(n, c)
                        return False
                    elif (c, n) not in self.links:
                        self.request(c, n)
                        return False
        return True                            
\end{lstlisting}
\subsection{Hauptmethode}
\begin{lstlisting}[frame=single]
def superstar():             
    path = sys.argv[1]          
    g = graph(path)             
    sf = star_finder(g)        
    sf.run()                  

if __name__ == '__main__':
    superstar()                                            

 
\end{lstlisting}
\end{document}
